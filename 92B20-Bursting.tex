\documentclass[12pt]{article}
\usepackage{pmmeta}
\pmcanonicalname{Bursting}
\pmcreated{2013-03-22 16:28:51}
\pmmodified{2013-03-22 16:28:51}
\pmowner{emi}{15656}
\pmmodifier{emi}{15656}
\pmtitle{bursting}
\pmrecord{7}{38648}
\pmprivacy{1}
\pmauthor{emi}{15656}
\pmtype{Definition}
\pmcomment{trigger rebuild}
\pmclassification{msc}{92B20}
\pmclassification{msc}{92C20}
\pmsynonym{burst}{Bursting}

% this is the default PlanetMath preamble.  as your knowledge
% of TeX increases, you will probably want to edit this, but
% it should be fine as is for beginners.

% almost certainly you want these
\usepackage{amssymb}
\usepackage{amsmath}
\usepackage{amsfonts}

% used for TeXing text within eps files
%\usepackage{psfrag}
% need this for including graphics (\includegraphics)
%\usepackage{graphicx}
% for neatly defining theorems and propositions
%\usepackage{amsthm}
% making logically defined graphics
%%%\usepackage{xypic}

% there are many more packages, add them here as you need them

% define commands here

\begin{document}
\PMlinkescapeword{action}
\PMlinkescapeword{potentials}
\PMlinkescapeword{triplet}
\PMlinkescapeword{quadruplet}

In neuroscience, {\bf bursting} denotes two or more action potentials (spikes) fired by a neuron, followed by a period of quiescence. A burst of two spikes is called a {\bf doublet}, three spikes - {\bf triplet}, four - {\bf quadruplet}, etc. 

Most mathematical models of bursting can be written in the singularly perturbed form
\[
\begin{matrix} 
\dot{x}  = \  f(x, y) & \ \ \ \ \ \ \ \mbox{(fast spiking)} \\ 
\dot{y}  =  \mu g(x, y) & \mbox{(slow modulation)}
\end{matrix}
\]
where $x \in {\mathbb R}^n$ is the fast variable that simulates fast spiking of the neuron, and $y \in {\mathbb R}^m$ is the slow variable that modulates such spiking activity.

A topological classification of bursters relies on the bifurcations of the fast subsystem (variable $x$) when the slow subsystem (variable $y$) is treated as a parameter.

\begin{thebibliography}{20}

\bibitem[iz]{IZ} Izhikevich E.M. (2007) Dynamical Systems in Neuroscience: The Geometry of Excitability and Bursting. The MIT Press.

\bibitem[izb]{IZB} Eugene M. Izhikevich (2006) Bursting. Scholarpedia, p.1401 (available online at http://www.scholarpedia.org/article/Bursting).

\end{thebibliography}


 
%%%%%
%%%%%
\end{document}
