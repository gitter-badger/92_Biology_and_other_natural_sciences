\documentclass[12pt]{article}
\usepackage{pmmeta}
\pmcanonicalname{LindenmayerSystem}
\pmcreated{2013-03-22 18:57:31}
\pmmodified{2013-03-22 18:57:31}
\pmowner{CWoo}{3771}
\pmmodifier{CWoo}{3771}
\pmtitle{Lindenmayer system}
\pmrecord{11}{41816}
\pmprivacy{1}
\pmauthor{CWoo}{3771}
\pmtype{Definition}
\pmcomment{trigger rebuild}
\pmclassification{msc}{92C80}
\pmclassification{msc}{68Q45}
\pmsynonym{L system}{LindenmayerSystem}
\pmsynonym{DL-system}{LindenmayerSystem}
\pmsynonym{PL-system}{LindenmayerSystem}
\pmdefines{L-system}
\pmdefines{start word}
\pmdefines{deterministic L-system}
\pmdefines{propagating L-system}
\pmdefines{L-language}
\pmdefines{EL-system}
\pmdefines{2L-system}
\pmdefines{1L-system}

\usepackage{amssymb,amscd}
\usepackage{amsmath}
\usepackage{amsfonts}
\usepackage{mathrsfs}

% used for TeXing text within eps files
%\usepackage{psfrag}
% need this for including graphics (\includegraphics)
%\usepackage{graphicx}
% for neatly defining theorems and propositions
\usepackage{amsthm}
% making logically defined graphics
%%\usepackage{xypic}
\usepackage{pst-plot}

% define commands here
\newcommand*{\abs}[1]{\left\lvert #1\right\rvert}
\newtheorem{prop}{Proposition}
\newtheorem{thm}{Theorem}
\newtheorem{ex}{Example}
\newcommand{\real}{\mathbb{R}}
\newcommand{\pdiff}[2]{\frac{\partial #1}{\partial #2}}
\newcommand{\mpdiff}[3]{\frac{\partial^#1 #2}{\partial #3^#1}}
\begin{document}
\subsubsection*{Definition}

Lindenmayer systems, or L-systems for short, are a variant of general rewriting systems.  Like a rewriting system, an L-system is also a language generator, where words are generated by applications of finite numbers of rewriting steps to some initial word given in advance.  However, unlike a rewriting system, rewriting occurs in \emph{parallel} in an L-system.  The notion of L-system was introduced by plant biologist Aristid Lindenmayer when he was studying the growth development of red algae.

Formally, an \emph{L-system} is a triple $G=(\Sigma,P,w)$, where 
\begin{enumerate}
\item $\Sigma$ is an alphabet, 
\item $w$ is a word over $\Sigma$, and 
\item $P$ is a finite subset of $\Sigma\times \Sigma^*$ such that for every $a\in \Sigma$, there is at least one $u\in \Sigma^*$ such that $(a,u)\in P$.  
\end{enumerate}
$w$ is called the \emph{start word}, or the \emph{axiom} of $G$, and elements of $P$ are called productions of the L-system $G$, and are written $a\to u$ instead of $(a,u)$.

As stated above, an L-system is a language generator, where words are generated from the axiom $w$ by repeated applications of productions of $P$.  Let us see how this is done.  Define a binary relation $\Rightarrow$ on $\Sigma^*$ as follows: for words $u,v\in \Sigma^*$, 
\begin{quote}
$u\Rightarrow v$ iff either $u=a_1\cdots a_n$ and $v=v_1\cdots v_n$, where $a_i\to v_i\in P$, or $u=v$.
\end{quote}
Now, take the transitive closure $\Rightarrow^*$ of $\Rightarrow$ and set 
$$L(G):=\lbrace u\mid w\Rightarrow^* u\rbrace.$$
Then $L(G)$ is called the \emph{language generated} by the L-system $G$.  An L-language is $L(G)$ for some L-system $G$.

\subsubsection*{Examples}
\begin{enumerate}
\item Let $G=(\lbrace a\rbrace, \lbrace a\to a^2\rbrace, a)$.  In two derivations, we get $a\Rightarrow a^2 \Rightarrow a^2a^2 = a^4$.  It is easy to see that after $n$ derivations, we get $a\Rightarrow^* a^{2^n}$, and that $L(G)=\lbrace a^{2^n} \mid n\ge 1\rbrace$.  Note that if parallel rewriting is not required then $a^3$ may be derived in three steps: $a\Rightarrow a^2 \Rightarrow (a^2)a=a^3$.
\item Let $G=(\lbrace a\rbrace, \lbrace a\to a, a\to a^2\rbrace, a)$.  Then $L(G)=\lbrace a\rbrace^+$.
\item Let $G=(\lbrace a\rbrace, \lbrace a\to \lambda, a\to a^2\rbrace, a)$.  Then $L(G)=\lbrace a^{2n}\mid n\ge 0\rbrace \cup \lbrace a\rbrace$.
\item Let $G=(\lbrace a,b\rbrace, \lbrace a\to ab, b\to ba\rbrace, a)$.  Then we get a sequence of words $a\Rightarrow ab \Rightarrow abba \Rightarrow abbabaab \Rightarrow \cdots $, and $L(G)$ is the set containing words in the sequence.  Note the recursive nature of the sequence: if $u_n$ is the $n$th word in the sequence, then $u_1 = a$ and $u_{n+1}=u_n h(u_n)$, where $h$ is the homomorphism given by $h(a)=b$ and $h(b)=a$.
\item L-systems can be used to generate graphs.  Usually, symbols in $\Sigma$ represent instructions on how to construct the graph.  For example, $$G=(\lbrace a,b,c\rbrace, \lbrace a\to a, b\to b, c\to cacbbcac \rbrace, c)$$ generates the famous Koch curve.  If $u \in L(G)$ is derived from $c$ in $n$ steps, then $u$ represents the $n$-th iteration of the Koch curve.  To draw the $n$-th iteration based on $u$, we do the following: 
\begin{enumerate}
\item write $u=d_1\cdots d_m$, where $d_i\in \Sigma$ (it is easy to see that $m=2^{n-1}$).
\item at each $d_i$, a current position $z_i$, and current direction $\theta_i$, are given.
\item start at the origin on the Euclidean plane in the positive $x$ direction, so that $z_0=(0,0)$ and $\theta_0=0$.
\item upon reading $d_i$, where $i>0$:
\begin{itemize}
\item if $d_i=a$, set $z_i = z_{i-1}$ and $\theta_i = \theta_{i-1}+60$,
\item if $d_i=b$, set $z_i = z_{i-1}$ and $\theta_i = \theta_{i-1}-60$,
\item if $d_i=c$, draw a line segment of unit length from $z_{i-1}$ to a point $P$ based on $\theta_{i-1}$, and set  $z_i=P$ and $\theta_i=\theta_{i-1}$.
\end{itemize}
\end{enumerate}
\end{enumerate}

A production $b\to u$ is said to correspond to $a\in \Sigma$ if $b=a$.  Both productions in Example 2 correspond to $a$.  A production is said to be a constant production if it has the form $a\to a$.  A symbol in $\Sigma$ is called a \emph{constant} if the only corresponding production is the constant production.  In the last example above, $a,b$ are both constants.  $a$ is not a constant in Example 2, even though it has a corresponding constant production.

\subsubsection*{Properties}

Given an L-system $G=(\Sigma,P,w)$, we can associate a function $f_G:\Sigma\to 2^{\Sigma^*}$ as follows: for each $a\in \Sigma$, set $$f_G(a):=\lbrace u\mid a\to u\in P\rbrace.$$  Then $f_G$ extends to a substitution $s_G:\Sigma^*\to 2^{\Sigma^*}$.  It is easy to see that $s_G(w)$ is just the set of words derivable from $w$ in one step: $s_G(w)=\lbrace u\mid w\Rightarrow u\rbrace$.  In fact, $$L(G)=\bigcup \lbrace s_G^n(w)\mid n\ge 0\rbrace,$$ where $s_G^0(w)=\lbrace w\rbrace$, and $s_G^{n+1}(w)=s_G(s_G^n(w))$.

In relation to languages described by the Chomsky hierarchy, we have the following results:
\begin{enumerate}
\item Every L-language is context-free.
\item If an L-system $G=(\Sigma,P,w)$ contains a constant production for each symbol in $\Sigma$, then $L(G)$ is context-free.
\item Denote the families of regular, context-free, context-sensitive, and L-languages by $\mathscr{R},\mathscr{F},\mathscr{S},\mathscr{L}$, and set $\mathscr{X}_1=\mathscr{R}$, $\mathscr{X}_2=\mathscr{F}-\mathscr{R}$, and $\mathscr{X}_2=\mathscr{S}-\mathscr{F}$.  Then $\mathscr{L}\cap \mathscr{X}_i$ and $\overline{\mathscr{L}}\cap \mathscr{X}_i$ are non-empty for $i=1,2,3$.  Here, $\overline{\mathscr{L}}$ is the complement of $\mathscr{L}$ in $2^{\Sigma^*}$, the family of all languages over $\Sigma$.
\end{enumerate}

\subsubsection*{Subsystems}

An L-system is said to be \emph{deterministic} if every symbol in $\Sigma$ has at most one (hence exactly one) production corresponding to it.  A deterministic L-system is also called a DL-system.  Examples 1,4,5 above are DL-systems.  For a DL-system, the associated substitution is a homomorphism, which means that for each $n\ge 0$, the set $s_G^n(w)$ is a singleton, so we get a unique sequence of words $w_0, w_1, \ldots$, such that $w_n\Rightarrow w_{n+1}$.  If $|w_n|<|w_{n+1}|$ for some $n$, then the word sequence is infinite.  In particular, if $w_n$ is a prefix of $w_{n+1}$ for all large enough $n$, and the lengths of the words have the property that $|w_n|=|w_{n+1}|$ implies $|w_m|<|w_{m+1}|$ for some $m>n$, then the DL-system defines a unique infinite word (by taking the union of all finite words).  In Example 4 above, the infinite word we obtain is the famous Thue-Morse sequence (an infinite word is an infinite sequence).

An L-system is said to be \emph{propagating} if no productions are of the form $a\to \lambda$. A propagating L-system is also called a PL-system.  All examples above, except 3, are propagating.  A DPL-system is a deterministic propagating L-system.  In a DPL-system, the lengths of the words in the corresponding sequence are non-decreasing, and one may classify DPL-systems by how fast these lengths grow.

\subsubsection*{Variations}

There are also ways one can extend the generative capacity of an L-system by generalizing some or all of the criteria defining an L-system.  Below are some:
\begin{enumerate}
\item Create a partition of $\Sigma=N\cup T$, the set $N$ of non-terminals and the set $T$ of terminals, so that only terminal words are allowed in $L(G)$.  Such a system is called an EL-system.  Formally, an EL-system is a 4-tuple $$H=(N,T,P,w)$$ such that $G_H=(N\cup T,P,w)$ is an L-system, and $L(H)=L(G_H)\cap T^*$.
\item Notice that the productions in an L-system are context-free in the sense that during a rewriting step, the rewriting of a symbol does not depend on the ``context'' of the symbol (its neighboring symbols).  This is the reason why an L-system is also known as a 0L-system.  We can generalize an 0L-system by permitting context-sensitivity in the productions.  If the rewriting of a symbol depends both on its left and right neighboring symbols, the resulting system is called a 2L-system.  On the other hand, a 1L-system is a system such that dependency is one-sided.  

Formally, a 2L-system is a quadruple $$(\Sigma, P, w,\sqcup).$$ Both $\Sigma$ and $w$ are defined as in an L-system.  $\sqcup$ is a symbol not in $\Sigma$, denoting a blank space.  $P$ is a subset of $\Sigma_1 \times \Sigma\times \Sigma_1 \times \Sigma^*$, where $\Sigma_1 = \Sigma \cup \lbrace \sqcup \rbrace$, such that for every $(a,b,c)\in \Sigma_1 \times \Sigma\times \Sigma_1$, there is a $u\in \Sigma^*$ such that $(a,b,c,u)\in P$.  Elements of $P$ are called productions, and are written $abc\to u$ instead of $(a,b,c,u)$.  Rewriting works as follows: the binary relation $\Rightarrow$ on $\Sigma^*$ called a rewriting step, is given by $u\Rightarrow v$ iff either $u=v$, or $u=a_1\cdots a_n$ and $v=v_1\cdots v_n$, such that 
\begin{enumerate}
\item $\sqcup a_1 a_2 \to v_1$,
\item $a_{i-1} a_i a_{i+1} \to v_i$ where $i=2,\cdots, n-1$, and
\item $a_{n-1}a_n\sqcup \to v_n$.  
\end{enumerate}
If $n=2$, then productions of the second form above do not apply.  If $n=1$, then $\sqcup a_1 \sqcup \to v_1$ are the only productions.

A 1L-system is then a 2L-system such that either, $abc \to u$ for some $c\in \Sigma_1$ implies $abd\to u$ for all $d\in \Sigma_1$, or $cab \to u$ for some $c\in \Sigma_1$ implies $dab \to u$ for all $d\in \Sigma_1$.

It is easy to see that an L-system is a 2L-system such that if $abc\to u$ for some $a,c\in \Sigma_1$, then $dbe\to u$ for all $d,e\in \Sigma_1$.
\item Allow the possibility that not all of the symbols may be rewritten.  This means that $u\Rightarrow v$ iff either $u=v$, or $u=a_1\cdots a_n$ and $v=v_1\cdots v_n$, and either $a_i=v_i$ or $a_i\to v_i \in P$.
\item Allow more than one axiom.  In other words, the single axiom word $w$ is replaced by a set $W$ of axioms.
\end{enumerate}

\begin{thebibliography}{9}
\bibitem{AS} A. Salomaa, {\em Formal Languages}, Academic Press, New York (1973).
\end{thebibliography}
%%%%%
%%%%%
\end{document}
