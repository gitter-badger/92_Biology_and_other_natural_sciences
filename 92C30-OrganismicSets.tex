\documentclass[12pt]{article}
\usepackage{pmmeta}
\pmcanonicalname{OrganismicSets}
\pmcreated{2013-03-22 18:11:20}
\pmmodified{2013-03-22 18:11:20}
\pmowner{bci1}{20947}
\pmmodifier{bci1}{20947}
\pmtitle{organismic sets}
\pmrecord{28}{40762}
\pmprivacy{1}
\pmauthor{bci1}{20947}
\pmtype{Topic}
\pmcomment{trigger rebuild}
\pmclassification{msc}{92C30}
\pmclassification{msc}{92B99}
\pmclassification{msc}{92B20}
\pmclassification{msc}{92B05}
\pmsynonym{lattices or sets of organismic structures}{OrganismicSets}
%\pmkeywords{relational biology}
%\pmkeywords{predicate logic}
%\pmkeywords{set and relations theory}
%\pmkeywords{mathematical biology}
%\pmkeywords{organismic development}
%\pmkeywords{organization of complex biosystems}
\pmrelated{OrganismicSet}
\pmrelated{NicolasRashevsky}
\pmrelated{AbstractRelationalBiology}
\pmrelated{GeneticNetsOrNetworks}
\pmrelated{ComplexSystemsBiology}
\pmrelated{OrganismicSetTheory}
\pmrelated{FunctionalBiology}
\pmdefines{organismic set}

% this is the default PlanetMath preamble.  

% almost certainly you want these
\usepackage{amssymb}
\usepackage{amsmath}
\usepackage{amsfonts}

% define commands here
\begin{document}
\subsection{Organismic Set Theory and Relational Biology} 

{\em Organismic sets} ($OS^n$) were defined by Nicolas Rashevsky as simple set-theoretical models of organization in living organisms at discrete integer or zero levels by means of sets of several distinct types or order beginning at the zeroth order, and having an upper limit as the fifth or sixth order of roganization. Thus, in the case of organismic sets of zero-th order, $S_0$, the elements correspond to genes, and a concrete $S_{0c}$ is defined as the set of all genes $[G_n]$ of a specific organism or organism type (understood as a stable biological species). 

Alternatively, $S_{0c}$ can be defined as a set representation of any organismic genome, $G_O$, consisting of the complete set of active, functional genes of an organism together with controlling genes/operons. The latter are then also considered together with inputs $e_i$ from the environment, as well as their activities $a_i$ and relations $R_{ij}$ among organismic set elements (genes in the case of $S_0$), where $i, j$ are indices in a countable, index set $I$. Thus, Rashevsky's organismic set (OS) theory is part of \PMlinkname{abstract relational biology}{AbstractRelationalBiology}, but unlike organismic networks, metabolic-replication systems and organismic supercategories, Rashevsky's OS are only endowed with the discrete topology, and are thus the simplest model whose only connectivities to organization are through the hierarchical lattice structure of the different types of OS, from
$G_O$ to ($OS^n$), with $n>1$.
At the next level of biological organization, cells are  considered as \emph{first order organismic sets}, $S_1$, whereas multi-cellular organisms are modeled by organismic sets of second order, $S_2$, whose `elements' are the first order organismic sets, or cells, $S_1$. Attempts were then made by Rashevsky to expand his theory of organismic sets to organizational models of human societies. Results from such studies of {\em relations} between sets were considered to be far more important than the numerical or {\em quantitative} aspects that play such important roles in physics and chemistry. A number of interesting results were obtained by means of standard (Boolean) logic predicates applied to organismic sets and their relations. Further details can be found in the publications listed below and the references cited therein. Subsequently, autopoietic theories have enlarged upon, and also extended, the application of organismic sets to biological and ecological systems.


\begin{thebibliography}{9}

\bibitem{Rashevsky1-yr1965}
Rashevsky, N.: 1965, The Representation of Organisms in Terms of
Predicates, \emph{Bulletin of Mathematical Biophysics} \textbf{27}: 477-491.

\bibitem{Rashevsky2-1969}
Rashevsky, N.: 1969, Outline of a Unified Approach to Physics, Biology and Sociology., \emph{Bulletin of Mathematical Biophysics} \textbf{31}: 159--198.

\end{thebibliography}
%%%%%
%%%%%
\end{document}
