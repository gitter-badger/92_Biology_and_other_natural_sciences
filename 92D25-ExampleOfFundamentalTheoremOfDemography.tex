\documentclass[12pt]{article}
\usepackage{pmmeta}
\pmcanonicalname{ExampleOfFundamentalTheoremOfDemography}
\pmcreated{2013-03-22 14:15:59}
\pmmodified{2013-03-22 14:15:59}
\pmowner{mathcam}{2727}
\pmmodifier{mathcam}{2727}
\pmtitle{example of fundamental theorem of demography}
\pmrecord{6}{35716}
\pmprivacy{1}
\pmauthor{mathcam}{2727}
\pmtype{Example}
\pmcomment{trigger rebuild}
\pmclassification{msc}{92D25}
\pmclassification{msc}{37A30}

\endmetadata

% this is the default PlanetMath preamble.  as your knowledge
% of TeX increases, you will probably want to edit this, but
% it should be fine as is for beginners.

% almost certainly you want these
\usepackage{amssymb}
\usepackage{amsmath}
\usepackage{amsfonts}

% used for TeXing text within eps files
%\usepackage{psfrag}
% need this for including graphics (\includegraphics)
%\usepackage{graphicx}
% for neatly defining theorems and propositions
%\usepackage{amsthm}
% making logically defined graphics
%%%\usepackage{xypic}

% there are many more packages, add them here as you need them

% define commands here
\begin{document}
\PMlinkescapeword{states}
\PMlinkescapeword{dominant}
\PMlinkescapeword{class}
\PMlinkescapeword{level}
\PMlinkescapeword{represents}
\PMlinkescapeword{fixed}

Assume a population with a \PMlinkescapetext{characteristic} (age, sex, etc.) that is described by a vector $x(t)$, where $x_1(t),\ldots,x_n(t)$ represents the number of individuals in the population who possess the characteristic at a level $1,\ldots,n$, at time $t$.

For example, consider age-groups, and assume $x_0(t)$ is the number of individuals in the population that are aged 0 to 1 year, $x_1(t)$ is the number of individuals aged 1 to 2 years, etc.

Suppose that the transition from one class to another is described by a matrix $A(t)$. In the case of age-groups, this matrix will for example describe mortality in a given age-group. This matrix, in the case of non deterministic modelling, will define a Markov chain.

The fundamental theorem of demography then states that if the matrix $A(t)$ satisfies the required properties, then the distribution vector $x(t)$ converges to the eigenvector associated to the dominant eigenvalue, \emph{regardless} of the behavior of the total population $\|x(t)\|$.

Hence, in the case of age-groups, the \emph{proportion} of individuals aged, say, 1 to 2 years, tends to a fixed value, even if the total population increases.
%%%%%
%%%%%
\end{document}
