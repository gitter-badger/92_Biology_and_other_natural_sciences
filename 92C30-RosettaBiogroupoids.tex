\documentclass[12pt]{article}
\usepackage{pmmeta}
\pmcanonicalname{RosettaBiogroupoids}
\pmcreated{2013-03-22 18:11:57}
\pmmodified{2013-03-22 18:11:57}
\pmowner{bci1}{20947}
\pmmodifier{bci1}{20947}
\pmtitle{Rosetta biogroupoids}
\pmrecord{48}{40776}
\pmprivacy{1}
\pmauthor{bci1}{20947}
\pmtype{Topic}
\pmcomment{trigger rebuild}
\pmclassification{msc}{92C30}
\pmclassification{msc}{92B99}
\pmclassification{msc}{92B20}
\pmclassification{msc}{92B05}
\pmsynonym{coevolution groupoids}{RosettaBiogroupoids}
\pmsynonym{social interaction group patterns}{RosettaBiogroupoids}
%\pmkeywords{organsimic Supercategories}
%\pmkeywords{ultra-complex systems}
\pmrelated{UltraComplexSystems}
\pmrelated{CategoricalOntology}
\pmrelated{ErdHosNumber}
\pmrelated{ComplexSystemsBiology}
\pmrelated{SupercategoriesOfComplexSystems}
\pmrelated{Biogroupoids}
\pmrelated{AbstractRelationalBiology}
\pmrelated{CWComplexDefinitionRelatedToSpinNetworksAndSpinFoams}
\pmrelated{HighlyComplexSystems}
\pmdefines{topology of social interactions}
\pmdefines{social networking}

\endmetadata

% this is the default PlanetMath preamble.  as your knowledge
% of TeX increases, you will probably want to edit this, but
% it should be fine as is for beginners.

% almost certainly you want these
\usepackage{amssymb}
\usepackage{amsmath}
\usepackage{amsfonts}

% used for TeXing text within eps files
%\usepackage{psfrag}
% need this for including graphics (\includegraphics)
%\usepackage{graphicx}
% for neatly defining theorems and propositions
%\usepackage{amsthm}
% making logically defined graphics
%%%\usepackage{xypic}

% there are many more packages, add them here as you need them

% define commands here
% this is the default PlanetMath preamble.  as your knowledge
% of TeX increases, you will probably want to edit this, but
% it should be fine as is for beginners.

% almost certainly you want these
\usepackage{amssymb}
\usepackage{amsmath}
\usepackage{amsfonts}

% used for TeXing text within eps files
%\usepackage{psfrag}
% need this for including graphics (\includegraphics)
%\usepackage{graphicx}
% for neatly defining theorems and propositions
%\usepackage{amsthm}
% making logically defined graphics
%%%\usepackage{xypic}

% there are many more packages, add them here as you need them

% define commands here
\usepackage{amsmath, amssymb, amsfonts, amsthm, amscd, latexsym, enumerate}
%%\usepackage{xypic}
\usepackage[mathscr]{eucal}

\setlength{\textwidth}{6.5in}
%\setlength{\textwidth}{16cm}
\setlength{\textheight}{9.0in}
%\setlength{\textheight}{24cm}

\hoffset=-.75in     %%ps format
%\hoffset=-1.0in     %%hp format
\voffset=-.4in


\theoremstyle{plain}
\newtheorem{lemma}{Lemma}[section]
\newtheorem{proposition}{Proposition}[section]
\newtheorem{theorem}{Theorem}[section]
\newtheorem{corollary}{Corollary}[section]

\theoremstyle{definition}
\newtheorem{definition}{Definition}[section]
\newtheorem{example}{Example}[section]
%\theoremstyle{remark}
\newtheorem{remark}{Remark}[section]
\newtheorem*{notation}{Notation}
\newtheorem*{claim}{Claim}

\renewcommand{\thefootnote}{\ensuremath{\fnsymbol{footnote}}}
\numberwithin{equation}{section}

\newcommand{\Ad}{{\rm Ad}}
\newcommand{\Aut}{{\rm Aut}}
\newcommand{\Cl}{{\rm Cl}}
\newcommand{\Co}{{\rm Co}}
\newcommand{\DES}{{\rm DES}}
\newcommand{\Diff}{{\rm Diff}}
\newcommand{\Dom}{{\rm Dom}}
\newcommand{\Hol}{{\rm Hol}}
\newcommand{\Mon}{{\rm Mon}}
\newcommand{\Hom}{{\rm Hom}}
\newcommand{\Ker}{{\rm Ker}}
\newcommand{\Ind}{{\rm Ind}}
\newcommand{\IM}{{\rm Im}}
\newcommand{\Is}{{\rm Is}}
\newcommand{\ID}{{\rm id}}
\newcommand{\GL}{{\rm GL}}
\newcommand{\Iso}{{\rm Iso}}
\newcommand{\rO}{{\rm O}}
\newcommand{\Sem}{{\rm Sem}}
\newcommand{\St}{{\rm St}}
\newcommand{\Sym}{{\rm Sym}}
\newcommand{\SU}{{\rm SU}}
\newcommand{\Tor}{{\rm Tor}}
\newcommand{\U}{{\rm U}}

\newcommand{\A}{\mathcal A}
\newcommand{\Ce}{\mathcal C}
\newcommand{\D}{\mathcal D}
\newcommand{\E}{\mathcal E}
\newcommand{\F}{\mathcal F}
\newcommand{\G}{\mathcal G}
\renewcommand{\H}{\mathcal H}
\renewcommand{\cL}{\mathcal L}
\newcommand{\Q}{\mathcal Q}
\newcommand{\R}{\mathcal R}
\newcommand{\cS}{\mathcal S}
\newcommand{\cU}{\mathcal U}
\newcommand{\W}{\mathcal W}

\newcommand{\bA}{\mathbb{A}}
\newcommand{\bB}{\mathbb{B}}
\newcommand{\bC}{\mathbb{C}}
\newcommand{\bD}{\mathbb{D}}
\newcommand{\bE}{\mathbb{E}}
\newcommand{\bF}{\mathbb{F}}
\newcommand{\bG}{\mathbb{G}}
\newcommand{\bK}{\mathbb{K}}
\newcommand{\bM}{\mathbb{M}}
\newcommand{\bN}{\mathbb{N}}
\newcommand{\bO}{\mathbb{O}}
\newcommand{\bP}{\mathbb{P}}
\newcommand{\bR}{\mathbb{R}}
\newcommand{\bV}{\mathbb{V}}
\newcommand{\bZ}{\mathbb{Z}}

\newcommand{\bfE}{\mathbf{E}}
\newcommand{\bfX}{\mathbf{X}}
\newcommand{\bfY}{\mathbf{Y}}
\newcommand{\bfZ}{\mathbf{Z}}

\renewcommand{\O}{\Omega}
\renewcommand{\o}{\omega}
\newcommand{\vp}{\varphi}
\newcommand{\vep}{\varepsilon}

\newcommand{\diag}{{\rm diag}}
\newcommand{\grp}{{\mathsf{G}}}
\newcommand{\dgrp}{{\mathsf{D}}}
\newcommand{\desp}{{\mathsf{D}^{\rm{es}}}}
\newcommand{\Geod}{{\rm Geod}}
\newcommand{\geod}{{\rm geod}}
\newcommand{\hgr}{{\mathsf{H}}}
\newcommand{\mgr}{{\mathsf{M}}}
\newcommand{\ob}{{\rm Ob}}
\newcommand{\obg}{{\rm Ob(\mathsf{G)}}}
\newcommand{\obgp}{{\rm Ob(\mathsf{G}')}}
\newcommand{\obh}{{\rm Ob(\mathsf{H})}}
\newcommand{\Osmooth}{{\Omega^{\infty}(X,*)}}
\newcommand{\ghomotop}{{\rho_2^{\square}}}
\newcommand{\gcalp}{{\mathsf{G}(\mathcal P)}}

\newcommand{\rf}{{R_{\mathcal F}}}
\newcommand{\glob}{{\rm glob}}
\newcommand{\loc}{{\rm loc}}
\newcommand{\TOP}{{\rm TOP}}

\newcommand{\wti}{\widetilde}
\newcommand{\what}{\widehat}

\renewcommand{\a}{\alpha}
\newcommand{\be}{\beta}
\newcommand{\ga}{\gamma}
\newcommand{\Ga}{\Gamma}
\newcommand{\de}{\delta}
\newcommand{\del}{\partial}
\newcommand{\ka}{\kappa}
\newcommand{\si}{\sigma}
\newcommand{\ta}{\tau}

\newcommand{\med}{\medbreak}
\newcommand{\medn}{\medbreak \noindent}
\newcommand{\bign}{\bigbreak \noindent}

\newcommand{\lra}{{\longrightarrow}}
\newcommand{\ra}{{\rightarrow}}
\newcommand{\rat}{{\rightarrowtail}}
\newcommand{\ovset}[1]{\overset {#1}{\ra}}
\newcommand{\ovsetl}[1]{\overset {#1}{\lra}}
\newcommand{\hr}{{\hookrightarrow}}
 \newcommand{\<}{{\langle}}

%\usepackage{geometry, amsmath,amssymb,latexsym,enumerate}
%%%\usepackage{xypic}

\def\baselinestretch{1.1}


\hyphenation{prod-ucts}

%\geometry{textwidth= 16 cm, textheight=21 cm}

\newcommand{\sqdiagram}[9]{$$ \diagram  #1  \rto^{#2} \dto_{#4}&
#3  \dto^{#5} \\ #6    \rto_{#7}  &  #8   \enddiagram
\eqno{\mbox{#9}}$$ }

\def\C{C^{\ast}}

\newcommand{\labto}[1]{\stackrel{#1}{\longrightarrow}}

%\newenvironment{proof}{\noindent {\bf Proof} }{ \hfill $\Box$
%{\mbox{}}

\newcommand{\quadr}[4]{\begin{pmatrix} & #1& \\[-1.1ex] #2 & & #3\\[-1.1ex]& #4&
 \end{pmatrix}}
\def\D{\mathsf{D}}
\begin{document}
\textbf{Background}
It seems that the awareness of the self of other individuals developed at first, and then,
through--and as \emph{as an extension of the others}-- to oneself,  \emph{self awareness} emerged in a final step. Such pre--historic, societal/social interactions that are based on consensus, are also called `mutual'. 
These considerations lead to a natural representation of one's `self' emergence in terms of the following mathematical concept.

\begin{definition}  
A \emph{Rosetta biogroupoid}, $\mathsf{R}_{\grp}$, is a representation of characteristic, star-connected
patterns of two--way social (cooperative) interactions: $\Longleftrightarrow$, or  $\Updownarrow$, that are defined as topological groupoids $\grp_S$ with star-- ($*$), or rose-- like, internal symmetries. From a strictly topological 
viewpoint, a \emph{Rosetta biogroupoid} is a particular type of highly connected network, or oriented multi--graph, consisting of central nodes (vertices) connected through bi-directional edges representing the mutual interactions
between individuals in a population or society; thus, it can also be regarded as a particular type of low-dimensional
\PMlinkname{CW complex with a (CW) cellular structure}{CWComplexDefinitionRelatedToSpinNetworksAndSpinFoams} determined by the organisms, such as humans, and their mutual interactions. 
\end{definition} 


\textbf{Example} An especially interesting case of Rosetta groupoids is that of \emph{freely generated structures over a graph}, or \emph{network}, of social interactions with different degrees/orders or levels of connectivities; a good example is that of the Erd\"os `connectivity' number, defined by the length of the chain connecting an Erd\"os co-author of a published mathematical paper with Erd\"os to another co-author of the first one, and so on. 

A \emph{`Rosetta biogroupoid'} structure can be depicted as in the following diagram, but possibly with as many as twenty five branches from the center, reference individual:
\begin{equation}
\xymatrix{ & \text{Neighbour's Self} \ar[d] &
\\  \text{Neighbour's Self} \ar[r] & {\text{Oneself}} \ar[r] \ar[l] \ar[u]\ar[d]  &
 \text{Neighbour's Self} \ar[l]
\\ &  \text{Neighbour's Self}   \ar[u] & }
\end{equation}

%.............................$Neighbor's.. SELF$
%\med
%...........................................$\Updownarrow$
%\med
%$Neighbor's.SELF \Longleftrightarrow 1-SELF\Longleftrightarrow Neighbor's. SELF$
%\med...........................................$\Updownarrow$
%\med
%$.............................$Neighbor's.. SELF$
\med
\noindent {\textbf{Diagram 1}}: \emph{A Rosetta biogroupoid} of
consensual, societal interactions leading to self-awareness, one's
self and full consciousness; there could be as few as five, or as
many as twenty five, individuals in a pre-historic society of
humans; here only four are represented as branches.




\textbf{Remark:} such cooperative interactions caused- through social and biological coevolution- the emergence of human (and/or `humanoid') structured languages with both syntax and semantics; the latter in their turn facilitated the development of self-awareness and the emergence of the human mind as an ultra-complex system, and consciousness as an ultra-complex (meta-process) of processes. 

%%%%%
%%%%%
\end{document}
